% Options for packages loaded elsewhere
\PassOptionsToPackage{unicode}{hyperref}
\PassOptionsToPackage{hyphens}{url}
%
\documentclass[
]{book}
\usepackage{lmodern}
\usepackage{amssymb,amsmath}
\usepackage{ifxetex,ifluatex}
\ifnum 0\ifxetex 1\fi\ifluatex 1\fi=0 % if pdftex
  \usepackage[T1]{fontenc}
  \usepackage[utf8]{inputenc}
  \usepackage{textcomp} % provide euro and other symbols
\else % if luatex or xetex
  \usepackage{unicode-math}
  \defaultfontfeatures{Scale=MatchLowercase}
  \defaultfontfeatures[\rmfamily]{Ligatures=TeX,Scale=1}
\fi
% Use upquote if available, for straight quotes in verbatim environments
\IfFileExists{upquote.sty}{\usepackage{upquote}}{}
\IfFileExists{microtype.sty}{% use microtype if available
  \usepackage[]{microtype}
  \UseMicrotypeSet[protrusion]{basicmath} % disable protrusion for tt fonts
}{}
\makeatletter
\@ifundefined{KOMAClassName}{% if non-KOMA class
  \IfFileExists{parskip.sty}{%
    \usepackage{parskip}
  }{% else
    \setlength{\parindent}{0pt}
    \setlength{\parskip}{6pt plus 2pt minus 1pt}}
}{% if KOMA class
  \KOMAoptions{parskip=half}}
\makeatother
\usepackage{xcolor}
\IfFileExists{xurl.sty}{\usepackage{xurl}}{} % add URL line breaks if available
\IfFileExists{bookmark.sty}{\usepackage{bookmark}}{\usepackage{hyperref}}
\hypersetup{
  pdftitle={A Minimal Book Example},
  pdfauthor={Yihui Xie},
  hidelinks,
  pdfcreator={LaTeX via pandoc}}
\urlstyle{same} % disable monospaced font for URLs
\usepackage{color}
\usepackage{fancyvrb}
\newcommand{\VerbBar}{|}
\newcommand{\VERB}{\Verb[commandchars=\\\{\}]}
\DefineVerbatimEnvironment{Highlighting}{Verbatim}{commandchars=\\\{\}}
% Add ',fontsize=\small' for more characters per line
\usepackage{framed}
\definecolor{shadecolor}{RGB}{248,248,248}
\newenvironment{Shaded}{\begin{snugshade}}{\end{snugshade}}
\newcommand{\AlertTok}[1]{\textcolor[rgb]{0.94,0.16,0.16}{#1}}
\newcommand{\AnnotationTok}[1]{\textcolor[rgb]{0.56,0.35,0.01}{\textbf{\textit{#1}}}}
\newcommand{\AttributeTok}[1]{\textcolor[rgb]{0.77,0.63,0.00}{#1}}
\newcommand{\BaseNTok}[1]{\textcolor[rgb]{0.00,0.00,0.81}{#1}}
\newcommand{\BuiltInTok}[1]{#1}
\newcommand{\CharTok}[1]{\textcolor[rgb]{0.31,0.60,0.02}{#1}}
\newcommand{\CommentTok}[1]{\textcolor[rgb]{0.56,0.35,0.01}{\textit{#1}}}
\newcommand{\CommentVarTok}[1]{\textcolor[rgb]{0.56,0.35,0.01}{\textbf{\textit{#1}}}}
\newcommand{\ConstantTok}[1]{\textcolor[rgb]{0.00,0.00,0.00}{#1}}
\newcommand{\ControlFlowTok}[1]{\textcolor[rgb]{0.13,0.29,0.53}{\textbf{#1}}}
\newcommand{\DataTypeTok}[1]{\textcolor[rgb]{0.13,0.29,0.53}{#1}}
\newcommand{\DecValTok}[1]{\textcolor[rgb]{0.00,0.00,0.81}{#1}}
\newcommand{\DocumentationTok}[1]{\textcolor[rgb]{0.56,0.35,0.01}{\textbf{\textit{#1}}}}
\newcommand{\ErrorTok}[1]{\textcolor[rgb]{0.64,0.00,0.00}{\textbf{#1}}}
\newcommand{\ExtensionTok}[1]{#1}
\newcommand{\FloatTok}[1]{\textcolor[rgb]{0.00,0.00,0.81}{#1}}
\newcommand{\FunctionTok}[1]{\textcolor[rgb]{0.00,0.00,0.00}{#1}}
\newcommand{\ImportTok}[1]{#1}
\newcommand{\InformationTok}[1]{\textcolor[rgb]{0.56,0.35,0.01}{\textbf{\textit{#1}}}}
\newcommand{\KeywordTok}[1]{\textcolor[rgb]{0.13,0.29,0.53}{\textbf{#1}}}
\newcommand{\NormalTok}[1]{#1}
\newcommand{\OperatorTok}[1]{\textcolor[rgb]{0.81,0.36,0.00}{\textbf{#1}}}
\newcommand{\OtherTok}[1]{\textcolor[rgb]{0.56,0.35,0.01}{#1}}
\newcommand{\PreprocessorTok}[1]{\textcolor[rgb]{0.56,0.35,0.01}{\textit{#1}}}
\newcommand{\RegionMarkerTok}[1]{#1}
\newcommand{\SpecialCharTok}[1]{\textcolor[rgb]{0.00,0.00,0.00}{#1}}
\newcommand{\SpecialStringTok}[1]{\textcolor[rgb]{0.31,0.60,0.02}{#1}}
\newcommand{\StringTok}[1]{\textcolor[rgb]{0.31,0.60,0.02}{#1}}
\newcommand{\VariableTok}[1]{\textcolor[rgb]{0.00,0.00,0.00}{#1}}
\newcommand{\VerbatimStringTok}[1]{\textcolor[rgb]{0.31,0.60,0.02}{#1}}
\newcommand{\WarningTok}[1]{\textcolor[rgb]{0.56,0.35,0.01}{\textbf{\textit{#1}}}}
\usepackage{longtable,booktabs}
% Correct order of tables after \paragraph or \subparagraph
\usepackage{etoolbox}
\makeatletter
\patchcmd\longtable{\par}{\if@noskipsec\mbox{}\fi\par}{}{}
\makeatother
% Allow footnotes in longtable head/foot
\IfFileExists{footnotehyper.sty}{\usepackage{footnotehyper}}{\usepackage{footnote}}
\makesavenoteenv{longtable}
\usepackage{graphicx,grffile}
\makeatletter
\def\maxwidth{\ifdim\Gin@nat@width>\linewidth\linewidth\else\Gin@nat@width\fi}
\def\maxheight{\ifdim\Gin@nat@height>\textheight\textheight\else\Gin@nat@height\fi}
\makeatother
% Scale images if necessary, so that they will not overflow the page
% margins by default, and it is still possible to overwrite the defaults
% using explicit options in \includegraphics[width, height, ...]{}
\setkeys{Gin}{width=\maxwidth,height=\maxheight,keepaspectratio}
% Set default figure placement to htbp
\makeatletter
\def\fps@figure{htbp}
\makeatother
\setlength{\emergencystretch}{3em} % prevent overfull lines
\providecommand{\tightlist}{%
  \setlength{\itemsep}{0pt}\setlength{\parskip}{0pt}}
\setcounter{secnumdepth}{5}
\usepackage{booktabs}
\usepackage[]{natbib}
\bibliographystyle{apalike}

\title{A Minimal Book Example}
\author{Yihui Xie}
\date{2020-09-27}

\begin{document}
\maketitle

{
\setcounter{tocdepth}{1}
\tableofcontents
}
\hypertarget{prerequisites}{%
\chapter{Prerequisites}\label{prerequisites}}

This is a \emph{sample} book written in \textbf{Markdown}. You can use anything that Pandoc's Markdown supports, e.g., a math equation \(a^2 + b^2 = c^2\).

The \textbf{bookdown} package can be installed from CRAN or Github:

\begin{Shaded}
\begin{Highlighting}[]
\KeywordTok{install.packages}\NormalTok{(}\StringTok{"bookdown"}\NormalTok{)}
\CommentTok{# or the development version}
\CommentTok{# devtools::install_github("rstudio/bookdown")}
\end{Highlighting}
\end{Shaded}

Remember each Rmd file contains one and only one chapter, and a chapter is defined by the first-level heading \texttt{\#}.

To compile this example to PDF, you need XeLaTeX. You are recommended to install TinyTeX (which includes XeLaTeX): \url{https://yihui.org/tinytex/}.

\hypertarget{load-data}{%
\chapter{Load data}\label{load-data}}

\hypertarget{csv}{%
\section{csv}\label{csv}}

\begin{Shaded}
\begin{Highlighting}[]
\KeywordTok{library}\NormalTok{(data.table)}
\NormalTok{infile <-}\StringTok{ 'input_file.csv'}
\NormalTok{df <-}\StringTok{ }\KeywordTok{fread}\NormalTok{(infile, }\DataTypeTok{data.table =}\NormalTok{ F)}
\end{Highlighting}
\end{Shaded}

\hypertarget{excel}{%
\section{excel}\label{excel}}

\hypertarget{general-excel}{%
\subsection{general excel}\label{general-excel}}

\begin{Shaded}
\begin{Highlighting}[]
\KeywordTok{library}\NormalTok{(xlsx)}

\CommentTok{# simply load}
\NormalTok{df <-}\StringTok{ }\NormalTok{xlsx}\OperatorTok{::}\KeywordTok{read.xlsx}\NormalTok{(infile, }\DataTypeTok{sheetIndex =} \DecValTok{1}\NormalTok{)}
\NormalTok{df <-}\StringTok{ }\NormalTok{xlsx}\OperatorTok{::}\KeywordTok{read.xlsx}\NormalTok{(infile, }\DataTypeTok{sheetName =} \StringTok{'SheetName'}\NormalTok{)}

\CommentTok{# row/col customized}
\NormalTok{df <-}\StringTok{ }\NormalTok{xlsx}\OperatorTok{::}\KeywordTok{read.xlsx}\NormalTok{(infile,}
                      \DataTypeTok{sheetIndex =} \DecValTok{1}\NormalTok{,}
                      \DataTypeTok{startRow =} \DecValTok{2}\NormalTok{,}
                      \DataTypeTok{colIndex =}\NormalTok{ excelCol2Num[}\KeywordTok{c}\NormalTok{(}\StringTok{'AA'}\NormalTok{,}\StringTok{'B'}\NormalTok{)], }\CommentTok{# helper vector below}
                      \DataTypeTok{endRow =} \DecValTok{99}\NormalTok{)}
\end{Highlighting}
\end{Shaded}

\hypertarget{excel-with-password}{%
\subsection{excel with password}\label{excel-with-password}}

\begin{Shaded}
\begin{Highlighting}[]
\KeywordTok{library}\NormalTok{(xlsx)}
\NormalTok{df <-}\StringTok{ }\KeywordTok{read.xlsx}\NormalTok{(file.xlsx,}
                \DataTypeTok{sheetIndex =} \DecValTok{1}\NormalTok{,}
                \DataTypeTok{password =} \StringTok{'password'}\NormalTok{)}
\end{Highlighting}
\end{Shaded}

\hypertarget{larger-excel}{%
\subsection{larger excel}\label{larger-excel}}

\begin{Shaded}
\begin{Highlighting}[]
\KeywordTok{library}\NormalTok{(openxlsx)}
\NormalTok{df <-}\StringTok{ }\NormalTok{openxlsx}\OperatorTok{::}\KeywordTok{read.xlsx}\NormalTok{(infile)}
\end{Highlighting}
\end{Shaded}

\hypertarget{excel-column-letter-to-index-number}{%
\subsection{excel column letter to index number}\label{excel-column-letter-to-index-number}}

\begin{Shaded}
\begin{Highlighting}[]
\NormalTok{excelCol2Num <-}\StringTok{ }\DecValTok{1}\OperatorTok{:}\DecValTok{702}
\KeywordTok{names}\NormalTok{(excelCol2Num) <-}\StringTok{ }\KeywordTok{do.call}\NormalTok{(paste0, }\KeywordTok{expand.grid}\NormalTok{(LETTERS, }\KeywordTok{c}\NormalTok{(}\StringTok{""}\NormalTok{, LETTERS))[,}\DecValTok{2}\OperatorTok{:}\DecValTok{1}\NormalTok{])}

\NormalTok{excelCol2Num[}\KeywordTok{c}\NormalTok{(}\StringTok{'AA'}\NormalTok{, }\StringTok{'B'}\NormalTok{)]}
\end{Highlighting}
\end{Shaded}

\hypertarget{header-name-keep-special-character}{%
\subsection{header name keep special character}\label{header-name-keep-special-character}}

\begin{Shaded}
\begin{Highlighting}[]
\KeywordTok{library}\NormalTok{(janitor)}
\NormalTok{df_raw <-}\StringTok{ }\KeywordTok{read.csv}\NormalTok{(infile, }\DataTypeTok{check.names =}\NormalTok{ F)}
\NormalTok{df <-}\StringTok{ }\KeywordTok{clean_names}\NormalTok{(df_raw)}

\NormalTok{ref_colnames <-}\StringTok{ }\KeywordTok{data.frame}\NormalTok{(}\DataTypeTok{original =} \KeywordTok{colnames}\NormalTok{(df_raw), }
                           \DataTypeTok{row.names =} \KeywordTok{make_clean_names}\NormalTok{(}\KeywordTok{colnames}\NormalTok{(df_raw)))}
\end{Highlighting}
\end{Shaded}

\hypertarget{txt-vector}{%
\section{txt vector}\label{txt-vector}}

\begin{Shaded}
\begin{Highlighting}[]
\NormalTok{v1 <-}\StringTok{ }\KeywordTok{scan}\NormalTok{(}\StringTok{'file.txt'}\NormalTok{, }\DataTypeTok{what =}\NormalTok{ character, }\DataTypeTok{sep =} \StringTok{'}\CharTok{\textbackslash{}n}\StringTok{'}\NormalTok{)}
\end{Highlighting}
\end{Shaded}

\hypertarget{explore-data}{%
\chapter{Explore Data}\label{explore-data}}

\hypertarget{columns-quick-view}{%
\section{columns quick view}\label{columns-quick-view}}

\begin{Shaded}
\begin{Highlighting}[]
\KeywordTok{str}\NormalTok{(df)}

\KeywordTok{apply}\NormalTok{(df, }\DecValTok{2}\NormalTok{, }\ControlFlowTok{function}\NormalTok{(x) }\KeywordTok{range}\NormalTok{(x, }\DataTypeTok{na.rm =}\NormalTok{ T))}
\end{Highlighting}
\end{Shaded}

\hypertarget{clean-data}{%
\chapter{clean data}\label{clean-data}}

\hypertarget{duplicates}{%
\section{duplicates}\label{duplicates}}

\begin{Shaded}
\begin{Highlighting}[]
\NormalTok{df <-}\StringTok{ }\NormalTok{df[ }\OperatorTok{!}\KeywordTok{duplicated}\NormalTok{(df), ]}
\end{Highlighting}
\end{Shaded}

\hypertarget{column-types}{%
\section{column types}\label{column-types}}

\begin{Shaded}
\begin{Highlighting}[]
\KeywordTok{library}\NormalTok{(dplyr)}
\NormalTok{df <-}\StringTok{ }\NormalTok{df }\OperatorTok\StringTok{ }\KeywordTok{mutate_all}\NormalTok{(as.character)}
\end{Highlighting}
\end{Shaded}

\hypertarget{string-manipulation}{%
\section{string manipulation}\label{string-manipulation}}

\hypertarget{replace}{%
\subsection{replace}\label{replace}}

\begin{Shaded}
\begin{Highlighting}[]
\NormalTok{df}\OperatorTok{$}\NormalTok{text <-}\StringTok{ }\KeywordTok{gsub}\NormalTok{(}\StringTok{'old'}\NormalTok{, }\StringTok{'new'}\NormalTok{, df}\OperatorTok{$}\NormalTok{text)}
\end{Highlighting}
\end{Shaded}

\hypertarget{split}{%
\subsection{split}\label{split}}

\begin{Shaded}
\begin{Highlighting}[]
\NormalTok{text_part1 <-}\StringTok{ }\KeywordTok{sapply}\NormalTok{(}\KeywordTok{strsplit}\NormalTok{(df}\OperatorTok{$}\NormalTok{text_full, }\StringTok{' '}\NormalTok{), }\StringTok{'['}\NormalTok{, }\DecValTok{1}\NormalTok{)}
\end{Highlighting}
\end{Shaded}

\hypertarget{merge}{%
\section{merge}\label{merge}}

\begin{Shaded}
\begin{Highlighting}[]
\NormalTok{df <-}\StringTok{ }\KeywordTok{merge}\NormalTok{(df1, df2, }\DataTypeTok{by=}\KeywordTok{c}\NormalTok{(}\StringTok{'id'}\NormalTok{), }\DataTypeTok{all.x =}\NormalTok{ T)}

\NormalTok{df <-}\StringTok{ }\NormalTok{df1 }\OperatorTok
\StringTok{  }\KeywordTok{left_join}\NormalTok{(., df2, }\DataTypeTok{by =} \KeywordTok{c}\NormalTok{(}\StringTok{'id'}\NormalTok{))}
\end{Highlighting}
\end{Shaded}

\hypertarget{select-column-by-name}{%
\section{select column by name}\label{select-column-by-name}}

\begin{Shaded}
\begin{Highlighting}[]
\NormalTok{df <-}\StringTok{ }\KeywordTok{subset}\NormalTok{(df, }\DataTypeTok{select =} \KeywordTok{c}\NormalTok{(id, gender))}

\NormalTok{df <-}\StringTok{ }\NormalTok{df }\OperatorTok
\StringTok{  }\KeywordTok{select}\NormalTok{(}\StringTok{'id'}\NormalTok{, }\StringTok{'age'}\NormalTok{)}
\end{Highlighting}
\end{Shaded}

\hypertarget{subset}{%
\chapter{subset}\label{subset}}

Some \emph{significant} applications are demonstrated in this chapter.

\hypertarget{example-one}{%
\section{Example one}\label{example-one}}

\hypertarget{example-two}{%
\section{Example two}\label{example-two}}

\hypertarget{output}{%
\chapter{output}\label{output}}

\hypertarget{csv}{%
\section{csv}\label{csv}}

\begin{Shaded}
\begin{Highlighting}[]
\KeywordTok{write.csv}\NormalTok{(df, }\DataTypeTok{row.names =}\NormalTok{ F)}
\end{Highlighting}
\end{Shaded}

  \bibliography{book.bib,packages.bib}

\end{document}
